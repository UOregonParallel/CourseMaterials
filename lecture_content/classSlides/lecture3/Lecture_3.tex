%%%%%%%%%%%%%%%%%%%%%%%%%%%%%%%%%%%%%%%%%%%%%%%%%%%%%%%%%%%%%%%%%%%%%%%%%%
% File: Lecture_3.tex
% Authors: James Kress
% Date: February 1, 2014
% Description: 
%%%%%%%%%%%%%%%%%%%%%%%%%%%%%%%%%%%%%%%%%%%%%%%%%%%%%%%%%%%%%%%%%%%%%%%%%%

%<<<<<<<<<<<<<<<<<<<<<<<<<<<<<<<<<<<<<<<<<<<<<<<<<<<<<<<<<<<<<<<<<<<<<<<<<<<<<%
% Document package information
%>>>>>>>>>>>>>>>>>>>>>>>>>>>>>>>>>>>>>>>>>>>>>>>>>>>>>>>>>>>>>>>>>>>>>>>>>>>>>%
\documentclass[xcolor=dvipsnames]{beamer} 
%%%%%%%%%%%%%%%%%%%%%%%%%%%%%%%%%%%%%%%%%%%%%%%%%%%%%%%%%%%%%%%%%%%%%%%%%%
% File: _TeXdefs.tex
% Author: James Kress
% Date: January 25, 2014
% Description: A tex file containing the \usepackage declarations, and
%			   other document critial style settings.
%%%%%%%%%%%%%%%%%%%%%%%%%%%%%%%%%%%%%%%%%%%%%%%%%%%%%%%%%%%%%%%%%%%%%%%%%%

%-----------Package imports
\usepackage{graphicx}
\usepackage{pgfpages}
\usepackage{tikz}
\usepackage{latexsym}
\usepackage{verbatim}
%//////////END package imports


%----------Style elements
\useoutertheme{infolines} 
\usetheme{Frankfurt} 
\usepackage{../theme/beamercolorthemeoregon}
\setbeamertemplate{sections/subsections in toc}[default]
\setbeamertemplate{footline}
{
\leavevmode%
  \hbox{%
  \begin{beamercolorbox}[wd=.3\paperwidth,ht=2.25ex,dp=.75ex,center]{institute in head/foot}%
    \usebeamerfont{institute in head/foot}\insertshortinstitute
  \end{beamercolorbox}%
    \begin{beamercolorbox}[wd=.4\paperwidth,ht=2.25ex,dp=.75ex,center]{title in head/foot}%
      \usebeamerfont{title in head/foot}\insertshorttitle
    \end{beamercolorbox}%
  \begin{beamercolorbox}[wd=.3\paperwidth,ht=2.25ex,dp=.75ex,center]{date in head/foot}%
    \usebeamerfont{date in head/foot}\insertshortdate\hspace*{3em}
    \insertframenumber{} / \inserttotalframenumber\hspace*{1ex}
  \end{beamercolorbox}}%
  \vskip0pt%
}
%/////////END style elements


%---------Command Declarations
\DeclareGraphicsExtensions{.pdf, .jpeg, .png, .jpg}
\graphicspath{ {../images/} }
\newcommand{\className}{\text{CIS 510}}
\newcommand{\departmentName}{\textit{Department of Computer and 
									Information Science \\ University of Oregon}}
%/////////END command declarations


%---------Setup pdf properties
\hypersetup{
	pdfusetitle=true,
    bookmarks=true,         	% show bookmarks bar?
    unicode=false,          	% non-Latin characters in Acrobat’s bookmarks
    pdftoolbar=true,        	% show Acrobat’s toolbar?
    pdfmenubar=true,        	% show Acrobat’s menu?
    pdffitwindow=false,     	% window fit to page when opened
    pdfstartview={Fit},   		% fits the width of the page to the window    
    pdfauthor={},     % author
    pdfsubject={Parallel Programming},   	% subject of the document
    pdfcreator={},   			% creator of the document
    pdfproducer={}, 			% producer of the document
    pdfkeywords={University of Oregon, parallel programming}, 
    pdfnewwindow=true,      	% links in new window
    colorlinks=true,       		% false: boxed links; true: colored links
    linkcolor=white,          	% color of internal links
    hidelinks,
    citecolor=green,        	% color of links to bibliography
    filecolor=magenta,      	% color of file links
    urlcolor=cyan,           	% color of external links
    linktoc=page,
    pageanchor = true
}
%//////////END setup pdf properties


%END ALL


%<<<<<<<<<<<<<<<<<<<<<<<<<<<<<<<<<<<<<<<<<<<<<<<<<<<<<<<<<<<<<<<<<<<<<<<<<<<<<%
% END Document package information
%>>>>>>>>>>>>>>>>>>>>>>>>>>>>>>>>>>>>>>>>>>>>>>>>>>>>>>>>>>>>>>>>>>>>>>>>>>>>>%

%=============================================================================%
% Beginning: Title Page Material
%=============================================================================%
\begin{document}
	\title[Gather, Scatter and Pack Patterns]{Parallel Control Patterns\\Data Reorganization}
	\author[]{\className}
	\institute[\className]{\departmentName}
	\date{} 

	\titlegraphic{\centering 
		$\vcenter{\hbox{\includegraphics[height=.31in,width=2.0in]{oregonLogo}}}$
	}

	\begin{frame}
		\maketitle
	\end{frame}
%-----------------------------------------------------------------------------%
% End: Title Page Material
%-----------------------------------------------------------------------------%


%=============================================================================%
% Section -> Gather
%=============================================================================%
\section{Gather} 

	\begin{frame} \frametitle{Table of Contents}
		%\tableofcontents
		%\tableofcontents[pausesections]
		\tableofcontents[currentsection]
	\end{frame} 
	
	\subsection{Overview}
	
		\begin{frame} \frametitle{Overview}
			\begin{itemize}
				\item Combination of a map with a random read
				\item Executes a number of independent random reads in parallel
			\end{itemize}
		\end{frame}
		
	\subsection{Special Cases}
	
		\subsection*{Shift}
		\begin{frame} \frametitle{Shift Pattern}
			\begin{itemize}
				\item \textbf{1D Array}: moves data to the left or right in memeory or to lower or higher locations (assuming locations are numbered left to right)
				\item \textbf{Multidimensional Array}: offsets data by different amounts in each dimension
			\end{itemize}
		\end{frame}
		
		\subsection*{Zip and Unzip}
		\begin{frame} \frametitle{Zip and Unzip Pattern}
			\begin{itemize}
				\item Zip interleaves data
				\item Unzip reverses a zip, extracing subarrays at certain offsets and strides from an input array
				\item Examples:
					\begin{itemize}
						\item Computation of transposes of multidimensional arrays
						\item Conversion of array of structures to structures of arrays
					\end{itemize}
			\end{itemize}
		\end{frame}
	
%=============================================================================%
% End: Section -> Gather
%=============================================================================%


%=============================================================================%
% Section -> Scatter
%=============================================================================%
\section{Scatter} 

	\begin{frame} \frametitle{Table of Contents}
		%\tableofcontents
		%\tableofcontents[pausesections]
		\tableofcontents[currentsection]
	\end{frame} 
	
	\subsection{Overview}
	
		\begin{frame} \frametitle{Overview}
			\begin{itemize}
				\item Similar to gather, but write locations are provided as input rather than read locations
				\item Unlike gather, scatter is ill-defined when duplicates (i.e. collisions) appear in the collection of locations 
				\item In the case of a collision, it is unclear what the result should be since multiple output values are specified for a single output location
			\end{itemize}
		\end{frame}
	
	\subsection{Collision Resolution Solutions}
	
		\subsection*{Atomic Scatter}
		\begin{frame} \frametitle{Atomic Scatter}
			\begin{itemize}
				\item Resolves collisions non-deterministically but atomically (i.e. no rule defined to determine which input items will be retained)
				\item Upon collision, one and only one of the values written to a location will be written in its entirety; all other values written to the same location will be discarded
				\item Examples:
					\begin{itemize}
						\item marking pairs in collision detection
						\item computing set intersection or union 
					\end{itemize}
				\item \textbf{insert figure 6.5}
			\end{itemize}
		\end{frame}
	
		\subsection*{Permutation Scatter}
		\begin{frame} \frametitle{Permutation Scatter}
			\begin{itemize}
				\item Makes collisions illegal
				\item Examples:
					\begin{itemize}
						\item FFT scrambling
						\item matrix/image transpose
						\item unpacking
					\end{itemize}
				\item \textbf{insert figure 6.6}
			\end{itemize}
		\end{frame}
	
		\subsection*{Merge Scatter}
		\begin{frame} \frametitle{Merge Scatter}
			\begin{itemize}
				\item Resolves collisions by combining values using associative and commutative operators
				\item Examples:
					\begin{itemize}
						\item Histogram implementation
						\item Computation of mutual information and entropy
						\item Database updates
					\end{itemize}
				\item \textbf{insert figure 6.7}
			\end{itemize}
		\end{frame}
	
		\subsection*{Priority Scatter}
		\begin{frame} \frametitle{Priority Scatter}
			\begin{itemize}
				\item Resolves collisions deterministically using priorities
				\item Every element in the input array is assigned a priority based on its position $\rightarrow$ deterministic
				\item Combine merge and priority scatter so that writes to a given location are guaranteed to be ordered $\rightarrow$ 3D graphics rendering
				\item \textbf{insert figure 6.8}
			\end{itemize}
		\end{frame}
%=============================================================================%
% End: Section -> Scatter
%=============================================================================%


%=============================================================================%
% Section -> Converting Scatter to Gather
%=============================================================================%
\section{Converting Scatter to Gather} 

	\begin{frame} \frametitle{Table of Contents}
		%\tableofcontents
		%\tableofcontents[pausesections]
		\tableofcontents[currentsection]
	\end{frame} 
%=============================================================================%
% End: Section -> Converting Scatter to Gather
%=============================================================================%


%=============================================================================%
% Section -> Pack
%=============================================================================%
\section{Pack} 

	\begin{frame} \frametitle{Table of Contents}
		%\tableofcontents
		%\tableofcontents[pausesections]
		\tableofcontents[currentsection]
	\end{frame} 
	
	
	\subsection{Overview}
	
		\begin{frame} \frametitle{Overview}
	
		\end{frame}
		
	\subsection{Unpack}
	
		\begin{frame} \frametitle{Unpack}
	
		\end{frame}
	
	\subsection{Split}
	
		\begin{frame} \frametitle{Split}
	
		\end{frame}
	
	\subsection{Unsplit}
	
		\begin{frame} \frametitle{Unsplit}
	
		\end{frame}
	
%=============================================================================%
% End: Section -> Pack
%=============================================================================%


%=============================================================================%
% Section -> Fusing Map and Pack
%=============================================================================%
\section{Fusing Map and Pack} 

	\begin{frame} \frametitle{Table of Contents}
		%\tableofcontents
		%\tableofcontents[pausesections]
		\tableofcontents[currentsection]
	\end{frame} 
	
	\subsection{Expand}
		\begin{frame} \frametitle{Expand}
		
		\end{frame}
%=============================================================================%
% End: Section -> Fusing Map and Pack
%=============================================================================%


%=============================================================================%
% Section -> Geometric Decomposition and Partition
%=============================================================================%
\section{Geometric Decomposition} 

	\begin{frame} \frametitle{Table of Contents}
		%\tableofcontents
		%\tableofcontents[pausesections]
		\tableofcontents[currentsection]
	\end{frame} 
%=============================================================================%
% End: Section -> Geometric Decomposition and Partition
%=============================================================================%

\end{document}

%END ALL

